\documentclass{amsart}

\usepackage[english]{babel}
\usepackage[utf8]{inputenc}
\usepackage{graphicx}
\usepackage{mathtools}
\usepackage{amsthm}
\usepackage{amsfonts}
\usepackage{hyperref}
\usepackage[singlelinecheck=false]{caption}
\usepackage[backend=biber,url=true,doi=true,eprint=false,style=alphabetic]{biblatex}
\usepackage{enumitem}
\usepackage[justification=centering]{caption}
\usepackage{indentfirst}
\usepackage{algorithm}
\usepackage{algpseudocode}

\addbibresource{references.bib}

\DeclareMathOperator*{\argmin}{arg\,min}
\DeclareMathOperator*{\argmax}{arg\,max}

\newcommand\defeq{\mathrel{\overset{\makebox[0pt]{\mbox{\normalfont\tiny\sffamily def}}}{=}}}

\algrenewcommand\algorithmicrequire{\textbf{Input}}
\algrenewcommand\algorithmicensure{\textbf{Output}}

\captionsetup[table]{labelsep=space}

\theoremstyle{plain}

\newtheorem*{definition}{Definition}
\newtheorem*{theorem}{Theorem}
\newtheorem*{proposition}{Proposition}

\newcommand{\set}[1]{\mathcal{#1}}
\newcommand{\pr}{\mathbb{P}}
\renewcommand{\implies}{\Rightarrow}

\setlength{\parskip}{1em}

\title[]{1. Introduction to Probabilistic Models}
\author[]{Renato Lui Geh\\NUSP\@: 8536030}

\begin{document}

\begin{abstract}
  In this document we present the fundamentals of probabilistic models. We show the origins of
  reasoning through generalized propositional logic and the reasons as to why probabilities have
  shown themselves so sucessful in representing uncertainty. We then introduce probability calculus
  and how to perform simple queries by enumeration of values. Important concepts of probability,
  such as belief updating, independence, random variables and conditional probability are formally
  presented.
  \vspace*{-2.5em}
\end{abstract}

\maketitle

\section{Reasoning and Uncertainty in Artificial Intelligence}

Before the advent of probability theory on uncertainty, it was believed that it was only possible
to represent knowledge through the use of logic and other neo-calculist techniques. Although logic
does have its praises, such as its computational ease and natural semantics that resemble human
reasoning, it is problematic when one tackles the issue of uncertainty. When it comes to
representing the world, propositional and first order logic must assume a binary reading of data:
true or false. Indeed the world is not so simple, and this vision is, at best, simplistic, since we
find that degrees of beliefs are present everywhere. For example, one can not say that all birds
fly, for it may have its wing broken. Of course we could enumerate all instances where the given
bird cannot fly (e.g.\ broken wing, bird species, etc), but this approach is exhausting and
unrealistic. This issue can be ameliorated (if not solved) by the introduction of probabilities as
a means to measure degrees of belief. We could then say that the bird has a 90\% chance of flying,
discarding the remaining 10\% as a sum of all probabilities of an infinite set of improbable events
that could prevent the bird from flying without having to enumerate each one of them. However the
introduction of probability as a direct method of computing uncertainty was viewed with skepticism
among AI researchers. Although not so welcomed at first, probability theory gained incredible
support in the last decades, with influential researchers such as Judea Pearl actively supporting
its use~\cite{pearl-1988}. In this report we first take a look at the formalisms of propositional
logic when representing knowledge, analysing the semantics, properties and relationships. Once
we are familiarized with the difficulties logic presents to uncertainty we introduce the basics
of probability theory and how to compute simple queries.

\section{Propositional Logic}

Be it through probabilities to represent degrees of belief or through Boolean values to represent
truths of events, one needs to define a language to describe knowledge formally. When it comes to
representing truths, an appropriate language is propositional logic. The atomic element in
propositional logic is the variable. A variable $X$ takes values in a domain $\Omega$. Such value
assignment is represented as the sentence $X=x$, for $x\in \Omega$. A sentence is defined as
follows.

\begin{itemize}
  \item A variable $X$ is a sentence.
  \item If $\alpha$ and $\beta$ are sentences, then $\alpha \wedge \beta$, $\alpha \vee \beta$ and
    $\neg \alpha$ are sentences.
\end{itemize}

The sentence $\alpha \wedge \beta$ represents disjunction, $\alpha \vee \beta$ is called
conjunction and $\neg \alpha$ negation. The symbols $\wedge$, $\vee$ and $\neg$ are called logical
connectives and are the main constructs of propositional logic. Valuations to a variable $X$ to a
value $x\in \Omega$ are described via a function $\nu$. We say that $\nu \models \phi$ if $\nu$
satisfies the sentence $\phi$. That is, the valuation of $\phi$ belongs to the domain $\Omega$ that
$\nu$ maps. If $\nu$ does not satisfy $\phi$, we say that $\nu \not\models \phi$. This means $\phi$
is not in the domain and therefore $\nu \models \neg \phi$. For conjunction, we say that $\nu
\models \phi \vee \psi \iff \nu \models \phi$ or $\nu \models \psi$. Similarly, $\nu \models \phi
\wedge \psi \iff \nu \models \phi$ and $\nu \models \psi$. By guaranteeing that negation,
conjunction and disjunction obey these rules, one defines a field of sets. A field of sets is a
pair $(\Omega,\mathcal{F})$ where $\Omega$ is the set of valuations and $\mathcal{F}$ is the set
of possible outcomes closed under negation, union (i.e.\ conjunction) and intersection (i.e.\
disjunction).

\begin{proposition} (Exercise 7)\\
  Let $\Omega$ be the set of all valuations and $\mathcal{F}$ be the set of events.
  \begin{equation*}
    \alpha \coloneqq \{\nu \in \Omega : \nu \models \phi\}\text{, when $\phi$ is a sentence.}
  \end{equation*}
  Then $(\Omega,\mathcal{F})$ is a field of sets.
\end{proposition}

\begin{proof}
  To prove $(\Omega,\mathcal{F})$ is a field of sets we must show that $\mathcal{F}$ is a non-empty
  set closed under union, intersection and complement. Let $\phi,\psi \in \Omega$, $\nu \models
  (\phi \vee \psi) \iff \nu \models \phi$ or $\nu \models \psi$. But from our earlier assumption,
  $\phi,\psi \in \Omega$; therefore $\mathcal{F}$ is closed under union. Similarly, $\nu \models
  (\phi \wedge \psi) \iff \nu \models \phi$ and $\nu \models \psi$. Since both sentences $\phi,\psi
  \in \Omega$ then $\mathcal{F}$ is closed under intersection. Finally, for an arbitrary $\phi$,
  $\nu \models \neg \phi \iff \nu \not\models \phi$. Assume $\phi \in \Omega$. Then $\phi$
  satisfies in $\Omega$ and therefore $\neg\phi$ does not satisfy lest we create a contradiction.
  Thus, by the definition of satisfiabilitiy $\nu \models \neg\phi$. Now consider $\nu \not\models
  \neg\phi$, then $\nu \models \neg(\neg\phi)$ and so $\neg\phi$ is in $\Omega$. Therefore,
  $\mathcal{F}$ is closed under complements.
\end{proof}

\section{Probability Calculus}



\newpage

\printbibliography[]

\end{document}
