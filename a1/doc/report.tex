\documentclass{amsart}

\usepackage[english]{babel}
\usepackage[utf8]{inputenc}
\usepackage{titlesec}
\usepackage{graphicx}
\usepackage{mathtools}
\usepackage{amsthm}
\usepackage{amsfonts}
\usepackage[top=1.0in,bottom=1.0in]{geometry}
\usepackage{hyperref}
\usepackage[singlelinecheck=false]{caption}
\usepackage[backend=biber,url=true,doi=true,eprint=false,style=alphabetic]{biblatex}
\usepackage{enumitem}
\usepackage[x11names, rgb]{xcolor}
\usepackage{tikz}
\usepackage[justification=centering]{caption}
\usepackage{indentfirst}
\usepackage{bookmark}
\usepackage{minted}
\usepackage{algorithm}
\usepackage{algpseudocode}

\usetikzlibrary{snakes,arrows,shapes}

\addbibresource{references.bib}

\newcommand\blfootnote[1]{%
  \begingroup
  \renewcommand\thefootnote{}\footnote{#1}%
  \addtocounter{footnote}{-1}%
  \endgroup
}

\DeclareMathOperator*{\argmin}{arg\,min}
\DeclareMathOperator*{\argmax}{arg\,max}

\newcommand\defeq{\mathrel{\overset{\makebox[0pt]{\mbox{\normalfont\tiny\sffamily def}}}{=}}}

\algrenewcommand\algorithmicrequire{\textbf{Input}}
\algrenewcommand\algorithmicensure{\textbf{Output}}

\titleformat{\section}
  {\normalfont\scshape\bfseries}{\thesection}{1em}{}
\titleformat{\subsection}
  {\normalfont\scshape\bfseries}{\thesubsection}{1em}{}
\titleformat{\paragraph}
  {\normalfont}{\theparagraph}{1em}{}
\titleformat{\subparagraph}
  {\normalfont}{\thesubparagraph}{1em}{}

\captionsetup[table]{labelsep=space}

\theoremstyle{plain}

\newtheorem*{definition}{Definition}
\newtheorem*{theorem}{Theorem}

\newcommand{\set}[1]{\mathcal{#1}}
\newcommand{\pr}{\mathbb{P}}
\renewcommand{\implies}{\Rightarrow}

\setlength{\parskip}{1em}

\title[]{1. Introduction to Probabilistic Models}
\author[]{Renato Lui Geh\\NUSP\@: 8536030}

\begin{document}

\maketitle

\begin{abstract}
  In this document we present the fundamentals of probabilistic models. We show the origins of
  reasoning through generalized propositional logic and the reasons as to why probabilities have
  shown themselves so sucessful in representing uncertainty. We then introduce probability calculus
  and how to perform query by enumeration of values. Important concepts of probability, such as
  belief updating, independence, random variables and conditional probability are formally
  presented.
\end{abstract}

\section{Reasoning and Uncertainty in Artificial Intelligence}

Before the advent of probability theory on uncertainty, it was believed that it was only possible
to represent knowledge through the use of logic and other neo-calculist techniques. Although logic
does have its praises, such as its computational ease and natural semantics that resemble human
reasoning, it is problematic when one tackles the issue of uncertainty. When it comes to
representing the world, propositional and first order logic must assume a binary reading of data:
true or false. Indeed the world is not so simple, and this vision is, at best, simplistic, since we
find that degrees of beliefs are present everywhere. For example, one can not say that all birds
fly, for it may have its wing broken. Of course we could enumerate all instances where the given
bird cannot fly (e.g.\ broken wing, bird species, etc), but this approach is exhausting and
unrealistic. This issue can be ameliorated (if not solved) by the introduction of probabilities as
a means to measure degrees of belief. We could then say that the bird has a 90\% chance of flying,
discarding the remaining 10\% as a sum of all probabilities of an infinite set of improbable events
that could prevent the bird from flying without having to enumerate each one of them. However the
introduction of probability as a direct method of computing uncertainty was viewed with skepticism
among AI researchers. Although not so welcomed at first, probability theory gained incredible
support in the last decades, with influential researchers such as Judea Pearl actively supporting
its use~\cite{pearl-1988}. In this report we first take a look at the formalisms of propositional
logic when representing knowledge, analysing the semantics, properties and relationships. Once
we are familiarized with the difficulties logic presents to uncertainty we introduce the basics
of probability theory and how to compute simple queries.

\section{Propositional Logic}



\newpage

\printbibliography[]

\end{document}
